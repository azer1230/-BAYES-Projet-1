% Options for packages loaded elsewhere
\PassOptionsToPackage{unicode}{hyperref}
\PassOptionsToPackage{hyphens}{url}
%
\documentclass[
]{article}
\usepackage{amsmath,amssymb}
\usepackage{iftex}
\ifPDFTeX
  \usepackage[T1]{fontenc}
  \usepackage[utf8]{inputenc}
  \usepackage{textcomp} % provide euro and other symbols
\else % if luatex or xetex
  \usepackage{unicode-math} % this also loads fontspec
  \defaultfontfeatures{Scale=MatchLowercase}
  \defaultfontfeatures[\rmfamily]{Ligatures=TeX,Scale=1}
\fi
\usepackage{lmodern}
\ifPDFTeX\else
  % xetex/luatex font selection
\fi
% Use upquote if available, for straight quotes in verbatim environments
\IfFileExists{upquote.sty}{\usepackage{upquote}}{}
\IfFileExists{microtype.sty}{% use microtype if available
  \usepackage[]{microtype}
  \UseMicrotypeSet[protrusion]{basicmath} % disable protrusion for tt fonts
}{}
\makeatletter
\@ifundefined{KOMAClassName}{% if non-KOMA class
  \IfFileExists{parskip.sty}{%
    \usepackage{parskip}
  }{% else
    \setlength{\parindent}{0pt}
    \setlength{\parskip}{6pt plus 2pt minus 1pt}}
}{% if KOMA class
  \KOMAoptions{parskip=half}}
\makeatother
\usepackage{xcolor}
\usepackage[margin=1in]{geometry}
\usepackage{graphicx}
\makeatletter
\def\maxwidth{\ifdim\Gin@nat@width>\linewidth\linewidth\else\Gin@nat@width\fi}
\def\maxheight{\ifdim\Gin@nat@height>\textheight\textheight\else\Gin@nat@height\fi}
\makeatother
% Scale images if necessary, so that they will not overflow the page
% margins by default, and it is still possible to overwrite the defaults
% using explicit options in \includegraphics[width, height, ...]{}
\setkeys{Gin}{width=\maxwidth,height=\maxheight,keepaspectratio}
% Set default figure placement to htbp
\makeatletter
\def\fps@figure{htbp}
\makeatother
\setlength{\emergencystretch}{3em} % prevent overfull lines
\providecommand{\tightlist}{%
  \setlength{\itemsep}{0pt}\setlength{\parskip}{0pt}}
\setcounter{secnumdepth}{-\maxdimen} % remove section numbering
\ifLuaTeX
  \usepackage{selnolig}  % disable illegal ligatures
\fi
\IfFileExists{bookmark.sty}{\usepackage{bookmark}}{\usepackage{hyperref}}
\IfFileExists{xurl.sty}{\usepackage{xurl}}{} % add URL line breaks if available
\urlstyle{same}
\hypersetup{
  pdftitle={Seeds - Random effect logistic regression},
  pdfauthor={BRUNO LOPES Matheus, MOURDI Elias, SELAMNIA Najib, TRIOMPHE Amaury},
  hidelinks,
  pdfcreator={LaTeX via pandoc}}

\title{Seeds - Random effect logistic regression}
\author{BRUNO LOPES Matheus, MOURDI Elias, SELAMNIA Najib, TRIOMPHE
Amaury}
\date{15/04/2024}

\begin{document}
\maketitle

\textbf{Lien vers notre Github} :
\texttt{https://github.com/azer1230/-BAYES-Projet-1}

\hypertarget{donnuxe9es-uxe9tudiuxe9es}{%
\section{Données étudiées}\label{donnuxe9es-uxe9tudiuxe9es}}

Le contexte de notre projet est l'étude de la germination de graines
respectant certaines propriétés. Dans notre exemple, \(N = 21\) plaques
sont disposées pour accueillir 2 types de graines issues de 2 types de
racines. Les tableaux ci-dessous recensent les résultat pour ces 4 types
de population. \(\forall i \in \{1,...,21\}, r_i\) correspond au nombre
de graines germées et \(n_i\) correspond au nombre total de graines sur
la \(i-\)ème plaque. Le rapport entre ces deux grandeurs est donc la
proportion de graines ayant germé sur la dite plaque.

\begin{table}[h]
\centering
\small
\begin{minipage}{0.45\textwidth}
\centering
\begin{tabular}{|c|c|c|c|c|c|c|c|}
\hline
\multicolumn{3}{|c|}{\textbf{Bean}} & \multicolumn{3}{|c|}{\textbf{Cucumber}} \\
\hline
\textbf{r} & \textbf{n} & \textbf{r/n} & \textbf{r} & \textbf{n} & \textbf{r/n} \\
\hline
10 & 39 & 0.26 & 5 & 6 & 0.83 \\
23 & 62 & 0.37 & 53 & 74 & 0.72 \\
23 & 81 & 0.28 & 55 & 72 & 0.76 \\
26 & 51 & 0.51 & 32 & 51 & 0.63 \\
17 & 39 & 0.44 & 46 & 79 & 0.58 \\
 & & & 10 & 13 & 0.77 \\
\hline
\end{tabular}
\caption{Données récupérées pour la graine seed O. aegyptiaco 75}
\label{tab:tableau1}
\end{minipage}\hfill
\begin{minipage}{0.45\textwidth}
\centering
\begin{tabular}{|c|c|c|c|c|c|c|c|}
\hline
\multicolumn{3}{|c|}{\textbf{Bean}} & \multicolumn{3}{|c|}{\textbf{Cucumber}} \\
\hline
\textbf{r} & \textbf{n} & \textbf{r/n} & \textbf{r} & \textbf{n} & \textbf{r/n} \\
\hline
8 & 16 & 0.5 & 3 & 12 & 0.25 \\
10 & 30 & 0.33 & 22 & 41 & 0.54 \\
8 & 28 & 0.29 & 15 & 30 & 0.5 \\
23 & 45 & 0.51 & 32 & 51 & 0.63 \\
0 & 4 & 0 & 3 & 7 & 0.43 \\
\hline
\end{tabular}
\caption{Données récupérées pour la graine seed O. aegyptiaco 73}
\label{tab:tableau2}
\end{minipage}
\end{table}

\hypertarget{cadre-mathuxe9matique}{%
\section{Cadre mathématique}\label{cadre-mathuxe9matique}}

\hypertarget{hypothuxe8ses-sur-nos-donnuxe9es}{%
\subsection{Hypothèses sur nos données
(?)}\label{hypothuxe8ses-sur-nos-donnuxe9es}}

\hypertarget{graphe-acyclique-orientuxe9-najib}{%
\subsection{Graphe acyclique orienté
(Najib)}\label{graphe-acyclique-orientuxe9-najib}}

\hypertarget{lois-conditionnelles}{%
\subsection{Lois conditionnelles (?)}\label{lois-conditionnelles}}

\hypertarget{ruxe9sultats-de-limpluxe9mentation-algorithmique}{%
\section{Résultats de l'implémentation
algorithmique}\label{ruxe9sultats-de-limpluxe9mentation-algorithmique}}

Grâce au calcul précédent des lois conditionnelles, nous avons pu
implémenter un algorithme de Hasting Within Gibbs pour estimer nos
paramètres. De la même manière que ce qui est donné dans l'énoncé, nous
avons généré \(10^4\) réalisations auxquelles nous avons retiré les 1000
premières, correspondant à la burnin period. Les résultats obtenus,
ainsi qu'une comparaison avec ce qui est donné dans l'énoncé, sont
mentionnés dans le tableau ce-dessous.

\begin{table}[h]
\centering
\small
\begin{minipage}{0.45\textwidth}
\centering
\begin{tabular}{|c|c|c|c|c|}
\hline
\multicolumn{1}{|c|}{} &
\multicolumn{2}{|c|}{\textbf{Moyenne}} & \multicolumn{2}{|c|}{\textbf{Écart-type}} \\
\hline
\textbf{Paramètres} & \textbf{Résultat} & \textbf{Énoncé} & \textbf{Résultat} & \textbf{Énoncé} \\
\hline
$\alpha_0$ & -0.5562 & -0.5525 & 0.1865 & 0.1852 \\
$\alpha_1$ & 0.0706 & 0.08382 & 0.3252 & 0.3031 \\
$\alpha_{12}$ & -0.8021 & -0.8165 & 0.4564 & 0.4109 \\
$\alpha_2$ & 1.3511 & 1.346 & 0.2745 & 0.2564 \\
$\sigma$ & 0.3198 & 0.267 & 0.0661 & 0.1471 \\
\hline
\end{tabular}
\caption{Résultats de notre algorithme Hastings within Gibbs}
\end{minipage}
\end{table}

\begin{itemize}
\tightlist
\item
  Allure des chaines de Markov (Matheus)
\item
  Allure des densités des chaines (Najib)
\end{itemize}

\hypertarget{analyse-des-ruxe9sultats}{%
\section{Analyse des résultats (?)}\label{analyse-des-ruxe9sultats}}

\end{document}
