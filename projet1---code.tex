% Options for packages loaded elsewhere
\PassOptionsToPackage{unicode}{hyperref}
\PassOptionsToPackage{hyphens}{url}
%
\documentclass[
]{article}
\usepackage{amsmath,amssymb}
\usepackage{iftex}
\ifPDFTeX
  \usepackage[T1]{fontenc}
  \usepackage[utf8]{inputenc}
  \usepackage{textcomp} % provide euro and other symbols
\else % if luatex or xetex
  \usepackage{unicode-math} % this also loads fontspec
  \defaultfontfeatures{Scale=MatchLowercase}
  \defaultfontfeatures[\rmfamily]{Ligatures=TeX,Scale=1}
\fi
\usepackage{lmodern}
\ifPDFTeX\else
  % xetex/luatex font selection
\fi
% Use upquote if available, for straight quotes in verbatim environments
\IfFileExists{upquote.sty}{\usepackage{upquote}}{}
\IfFileExists{microtype.sty}{% use microtype if available
  \usepackage[]{microtype}
  \UseMicrotypeSet[protrusion]{basicmath} % disable protrusion for tt fonts
}{}
\makeatletter
\@ifundefined{KOMAClassName}{% if non-KOMA class
  \IfFileExists{parskip.sty}{%
    \usepackage{parskip}
  }{% else
    \setlength{\parindent}{0pt}
    \setlength{\parskip}{6pt plus 2pt minus 1pt}}
}{% if KOMA class
  \KOMAoptions{parskip=half}}
\makeatother
\usepackage{xcolor}
\usepackage[margin=1in]{geometry}
\usepackage{color}
\usepackage{fancyvrb}
\newcommand{\VerbBar}{|}
\newcommand{\VERB}{\Verb[commandchars=\\\{\}]}
\DefineVerbatimEnvironment{Highlighting}{Verbatim}{commandchars=\\\{\}}
% Add ',fontsize=\small' for more characters per line
\usepackage{framed}
\definecolor{shadecolor}{RGB}{248,248,248}
\newenvironment{Shaded}{\begin{snugshade}}{\end{snugshade}}
\newcommand{\AlertTok}[1]{\textcolor[rgb]{0.94,0.16,0.16}{#1}}
\newcommand{\AnnotationTok}[1]{\textcolor[rgb]{0.56,0.35,0.01}{\textbf{\textit{#1}}}}
\newcommand{\AttributeTok}[1]{\textcolor[rgb]{0.13,0.29,0.53}{#1}}
\newcommand{\BaseNTok}[1]{\textcolor[rgb]{0.00,0.00,0.81}{#1}}
\newcommand{\BuiltInTok}[1]{#1}
\newcommand{\CharTok}[1]{\textcolor[rgb]{0.31,0.60,0.02}{#1}}
\newcommand{\CommentTok}[1]{\textcolor[rgb]{0.56,0.35,0.01}{\textit{#1}}}
\newcommand{\CommentVarTok}[1]{\textcolor[rgb]{0.56,0.35,0.01}{\textbf{\textit{#1}}}}
\newcommand{\ConstantTok}[1]{\textcolor[rgb]{0.56,0.35,0.01}{#1}}
\newcommand{\ControlFlowTok}[1]{\textcolor[rgb]{0.13,0.29,0.53}{\textbf{#1}}}
\newcommand{\DataTypeTok}[1]{\textcolor[rgb]{0.13,0.29,0.53}{#1}}
\newcommand{\DecValTok}[1]{\textcolor[rgb]{0.00,0.00,0.81}{#1}}
\newcommand{\DocumentationTok}[1]{\textcolor[rgb]{0.56,0.35,0.01}{\textbf{\textit{#1}}}}
\newcommand{\ErrorTok}[1]{\textcolor[rgb]{0.64,0.00,0.00}{\textbf{#1}}}
\newcommand{\ExtensionTok}[1]{#1}
\newcommand{\FloatTok}[1]{\textcolor[rgb]{0.00,0.00,0.81}{#1}}
\newcommand{\FunctionTok}[1]{\textcolor[rgb]{0.13,0.29,0.53}{\textbf{#1}}}
\newcommand{\ImportTok}[1]{#1}
\newcommand{\InformationTok}[1]{\textcolor[rgb]{0.56,0.35,0.01}{\textbf{\textit{#1}}}}
\newcommand{\KeywordTok}[1]{\textcolor[rgb]{0.13,0.29,0.53}{\textbf{#1}}}
\newcommand{\NormalTok}[1]{#1}
\newcommand{\OperatorTok}[1]{\textcolor[rgb]{0.81,0.36,0.00}{\textbf{#1}}}
\newcommand{\OtherTok}[1]{\textcolor[rgb]{0.56,0.35,0.01}{#1}}
\newcommand{\PreprocessorTok}[1]{\textcolor[rgb]{0.56,0.35,0.01}{\textit{#1}}}
\newcommand{\RegionMarkerTok}[1]{#1}
\newcommand{\SpecialCharTok}[1]{\textcolor[rgb]{0.81,0.36,0.00}{\textbf{#1}}}
\newcommand{\SpecialStringTok}[1]{\textcolor[rgb]{0.31,0.60,0.02}{#1}}
\newcommand{\StringTok}[1]{\textcolor[rgb]{0.31,0.60,0.02}{#1}}
\newcommand{\VariableTok}[1]{\textcolor[rgb]{0.00,0.00,0.00}{#1}}
\newcommand{\VerbatimStringTok}[1]{\textcolor[rgb]{0.31,0.60,0.02}{#1}}
\newcommand{\WarningTok}[1]{\textcolor[rgb]{0.56,0.35,0.01}{\textbf{\textit{#1}}}}
\usepackage{graphicx}
\makeatletter
\def\maxwidth{\ifdim\Gin@nat@width>\linewidth\linewidth\else\Gin@nat@width\fi}
\def\maxheight{\ifdim\Gin@nat@height>\textheight\textheight\else\Gin@nat@height\fi}
\makeatother
% Scale images if necessary, so that they will not overflow the page
% margins by default, and it is still possible to overwrite the defaults
% using explicit options in \includegraphics[width, height, ...]{}
\setkeys{Gin}{width=\maxwidth,height=\maxheight,keepaspectratio}
% Set default figure placement to htbp
\makeatletter
\def\fps@figure{htbp}
\makeatother
\setlength{\emergencystretch}{3em} % prevent overfull lines
\providecommand{\tightlist}{%
  \setlength{\itemsep}{0pt}\setlength{\parskip}{0pt}}
\setcounter{secnumdepth}{-\maxdimen} % remove section numbering
\ifLuaTeX
  \usepackage{selnolig}  % disable illegal ligatures
\fi
\IfFileExists{bookmark.sty}{\usepackage{bookmark}}{\usepackage{hyperref}}
\IfFileExists{xurl.sty}{\usepackage{xurl}}{} % add URL line breaks if available
\urlstyle{same}
\hypersetup{
  pdftitle={Seeds - Random effect logistic regression},
  pdfauthor={BRUNO LOPES Matheus, MOURDI Elias, SELAMNIA Najib, TRIOMPHE Amaury},
  hidelinks,
  pdfcreator={LaTeX via pandoc}}

\title{Seeds - Random effect logistic regression}
\author{BRUNO LOPES Matheus, MOURDI Elias, SELAMNIA Najib, TRIOMPHE
Amaury}
\date{15/04/2024}

\begin{document}
\maketitle

\hypertarget{importations-pruxe9liminaires}{%
\section{Importations
préliminaires}\label{importations-pruxe9liminaires}}

\begin{Shaded}
\begin{Highlighting}[]
\CommentTok{\# mettre ce qui nous sert pour la suite}
\end{Highlighting}
\end{Shaded}

\hypertarget{ruxe9cupuxe9ration-des-donnuxe9es}{%
\section{Récupération des
données}\label{ruxe9cupuxe9ration-des-donnuxe9es}}

\begin{Shaded}
\begin{Highlighting}[]
\CommentTok{\# Notations changées par rapport au fichier seeds.data pour être en accord avec }
\CommentTok{\# la consigne donnée}
\StringTok{"N"} \OtherTok{\textless{}{-}} \DecValTok{21} 
\StringTok{"r"} \OtherTok{\textless{}{-}} \FunctionTok{c}\NormalTok{(}\DecValTok{10}\NormalTok{, }\DecValTok{23}\NormalTok{, }\DecValTok{23}\NormalTok{, }\DecValTok{26}\NormalTok{, }\DecValTok{17}\NormalTok{, }\DecValTok{5}\NormalTok{, }\DecValTok{53}\NormalTok{, }\DecValTok{55}\NormalTok{, }\DecValTok{32}\NormalTok{, }\DecValTok{46}\NormalTok{, }\DecValTok{10}\NormalTok{, }\DecValTok{8}\NormalTok{, }\DecValTok{10}\NormalTok{, }\DecValTok{8}\NormalTok{, }\DecValTok{23}\NormalTok{, }\DecValTok{0}\NormalTok{, }\DecValTok{3}\NormalTok{, }\DecValTok{22}\NormalTok{, }\DecValTok{15}\NormalTok{, }\DecValTok{32}\NormalTok{, }\DecValTok{3}\NormalTok{)}
\StringTok{"n"} \OtherTok{\textless{}{-}} \FunctionTok{c}\NormalTok{(}\DecValTok{39}\NormalTok{, }\DecValTok{62}\NormalTok{, }\DecValTok{81}\NormalTok{, }\DecValTok{51}\NormalTok{, }\DecValTok{39}\NormalTok{, }\DecValTok{6}\NormalTok{, }\DecValTok{74}\NormalTok{, }\DecValTok{72}\NormalTok{, }\DecValTok{51}\NormalTok{, }\DecValTok{79}\NormalTok{, }\DecValTok{13}\NormalTok{, }\DecValTok{16}\NormalTok{, }\DecValTok{30}\NormalTok{, }\DecValTok{28}\NormalTok{, }\DecValTok{45}\NormalTok{, }\DecValTok{4}\NormalTok{, }\DecValTok{12}\NormalTok{, }\DecValTok{41}\NormalTok{, }\DecValTok{30}\NormalTok{, }\DecValTok{51}\NormalTok{, }\DecValTok{7}\NormalTok{)}
\StringTok{"x1"} \OtherTok{\textless{}{-}} \FunctionTok{c}\NormalTok{(}\DecValTok{0}\NormalTok{, }\DecValTok{0}\NormalTok{, }\DecValTok{0}\NormalTok{, }\DecValTok{0}\NormalTok{, }\DecValTok{0}\NormalTok{, }\DecValTok{0}\NormalTok{, }\DecValTok{0}\NormalTok{, }\DecValTok{0}\NormalTok{, }\DecValTok{0}\NormalTok{, }\DecValTok{0}\NormalTok{, }\DecValTok{0}\NormalTok{, }\DecValTok{1}\NormalTok{, }\DecValTok{1}\NormalTok{, }\DecValTok{1}\NormalTok{, }\DecValTok{1}\NormalTok{, }\DecValTok{1}\NormalTok{, }\DecValTok{1}\NormalTok{, }\DecValTok{1}\NormalTok{, }\DecValTok{1}\NormalTok{, }\DecValTok{1}\NormalTok{, }\DecValTok{1}\NormalTok{)}
\StringTok{"x2"} \OtherTok{\textless{}{-}} \FunctionTok{c}\NormalTok{(}\DecValTok{0}\NormalTok{, }\DecValTok{0}\NormalTok{, }\DecValTok{0}\NormalTok{, }\DecValTok{0}\NormalTok{, }\DecValTok{0}\NormalTok{, }\DecValTok{1}\NormalTok{, }\DecValTok{1}\NormalTok{, }\DecValTok{1}\NormalTok{, }\DecValTok{1}\NormalTok{, }\DecValTok{1}\NormalTok{, }\DecValTok{1}\NormalTok{, }\DecValTok{0}\NormalTok{, }\DecValTok{0}\NormalTok{, }\DecValTok{0}\NormalTok{, }\DecValTok{0}\NormalTok{, }\DecValTok{0}\NormalTok{, }\DecValTok{1}\NormalTok{, }\DecValTok{1}\NormalTok{, }\DecValTok{1}\NormalTok{, }\DecValTok{1}\NormalTok{, }\DecValTok{1}\NormalTok{)}
\end{Highlighting}
\end{Shaded}

\hypertarget{initialisation}{%
\section{Initialisation}\label{initialisation}}

\begin{Shaded}
\begin{Highlighting}[]
\NormalTok{alpha0 }\OtherTok{\textless{}{-}} \DecValTok{0}
\NormalTok{alpha1 }\OtherTok{\textless{}{-}} \DecValTok{0}
\NormalTok{alpha2 }\OtherTok{\textless{}{-}} \DecValTok{0}
\NormalTok{alpha12 }\OtherTok{\textless{}{-}} \DecValTok{0}
\NormalTok{tau }\OtherTok{\textless{}{-}} \DecValTok{10}
\NormalTok{b }\OtherTok{\textless{}{-}} \FunctionTok{c}\NormalTok{(}\DecValTok{0}\NormalTok{, }\DecValTok{0}\NormalTok{, }\DecValTok{0}\NormalTok{, }\DecValTok{0}\NormalTok{, }\DecValTok{0}\NormalTok{, }\DecValTok{0}\NormalTok{, }\DecValTok{0}\NormalTok{, }\DecValTok{0}\NormalTok{, }\DecValTok{0}\NormalTok{, }\DecValTok{0}\NormalTok{, }\DecValTok{0}\NormalTok{, }\DecValTok{0}\NormalTok{, }\DecValTok{0}\NormalTok{, }\DecValTok{0}\NormalTok{, }\DecValTok{0}\NormalTok{, }\DecValTok{0}\NormalTok{, }\DecValTok{0}\NormalTok{, }\DecValTok{0}\NormalTok{, }\DecValTok{0}\NormalTok{, }\DecValTok{0}\NormalTok{, }\DecValTok{0}\NormalTok{)}
\end{Highlighting}
\end{Shaded}

\hypertarget{impluxe9mentation-de-la-muxe9thode-mcmc-hastings-within-gibbs}{%
\section{Implémentation de la méthode MCMC (Hastings within
Gibbs)}\label{impluxe9mentation-de-la-muxe9thode-mcmc-hastings-within-gibbs}}

\begin{Shaded}
\begin{Highlighting}[]
\NormalTok{seeds }\OtherTok{=} \ControlFlowTok{function}\NormalTok{(nchain, N, r, n, x1, x2, alpha0, alpha1, alpha2, alpha12, tau, b, prop\_sd)\{}
  
  \CommentTok{\# Initialisation}
\NormalTok{  res }\OtherTok{=} \FunctionTok{matrix}\NormalTok{(}\ConstantTok{NA}\NormalTok{, nchain }\SpecialCharTok{+} \DecValTok{1}\NormalTok{, }\DecValTok{5}\NormalTok{)}
\NormalTok{  res[}\DecValTok{1}\NormalTok{, ] }\OtherTok{=} \FunctionTok{c}\NormalTok{(alpha0, alpha1, alpha12, alpha2, }\DecValTok{1}\SpecialCharTok{/}\FunctionTok{sqrt}\NormalTok{(tau)) }\CommentTok{\# le dernier terme vaut sigma}

\NormalTok{  p }\OtherTok{=} \FunctionTok{plogis}\NormalTok{(alpha0 }\SpecialCharTok{+}\NormalTok{ alpha1 }\SpecialCharTok{*}\NormalTok{ x1 }\SpecialCharTok{+}\NormalTok{ alpha2 }\SpecialCharTok{*}\NormalTok{ x2 }\SpecialCharTok{+}\NormalTok{ alpha12 }\SpecialCharTok{*}\NormalTok{ x1 }\SpecialCharTok{*}\NormalTok{ x2 }\SpecialCharTok{+}\NormalTok{ b)}
  
\NormalTok{  res\_b }\OtherTok{=} \FunctionTok{matrix}\NormalTok{(}\ConstantTok{NA}\NormalTok{, nchain }\SpecialCharTok{+} \DecValTok{1}\NormalTok{, N)}
\NormalTok{  res\_b[}\DecValTok{1}\NormalTok{, ] }\OtherTok{=}\NormalTok{ b}
  
\NormalTok{  acc\_rates }\OtherTok{=} \FunctionTok{rep}\NormalTok{(}\DecValTok{0}\NormalTok{, }\DecValTok{4}\NormalTok{) }\CommentTok{\# pour les 4 alpha}
  
  \ControlFlowTok{for}\NormalTok{ (i }\ControlFlowTok{in} \DecValTok{1}\SpecialCharTok{:}\NormalTok{nchain)\{}
    
    \CommentTok{\# Mise à jour de alpha0}
\NormalTok{    alpha0 }\OtherTok{=}\NormalTok{ res[i, }\DecValTok{1}\NormalTok{]}
\NormalTok{    prop }\OtherTok{=} \FunctionTok{rnorm}\NormalTok{(}\DecValTok{1}\NormalTok{, alpha0, prop\_sd[}\DecValTok{1}\NormalTok{]) }\CommentTok{\# marche aléatoire simple}
\NormalTok{    prop\_p }\OtherTok{=} \FunctionTok{plogis}\NormalTok{(prop }\SpecialCharTok{+}\NormalTok{ alpha1 }\SpecialCharTok{*}\NormalTok{ x1 }\SpecialCharTok{+}\NormalTok{ alpha2 }\SpecialCharTok{*}\NormalTok{ x2 }\SpecialCharTok{+}\NormalTok{ alpha12 }\SpecialCharTok{*}\NormalTok{ x1 }\SpecialCharTok{*}\NormalTok{ x2 }\SpecialCharTok{+}\NormalTok{ b)}
    
\NormalTok{    top }\OtherTok{=} \SpecialCharTok{{-}}\NormalTok{ ((prop}\SpecialCharTok{\^{}}\DecValTok{2}\NormalTok{) }\SpecialCharTok{/}\NormalTok{ (}\DecValTok{2} \SpecialCharTok{*} \FloatTok{1e6}\NormalTok{)) }\SpecialCharTok{+} \FunctionTok{sum}\NormalTok{(r }\SpecialCharTok{*} \FunctionTok{log}\NormalTok{(prop\_p)) }\SpecialCharTok{+} \FunctionTok{sum}\NormalTok{((n}\SpecialCharTok{{-}}\NormalTok{r) }\SpecialCharTok{*} \FunctionTok{log}\NormalTok{(}\DecValTok{1} \SpecialCharTok{{-}}\NormalTok{ prop\_p))}
\NormalTok{    bottom }\OtherTok{=} \SpecialCharTok{{-}}\NormalTok{ ((alpha0}\SpecialCharTok{\^{}}\DecValTok{2}\NormalTok{) }\SpecialCharTok{/}\NormalTok{ (}\DecValTok{2} \SpecialCharTok{*} \FloatTok{1e6}\NormalTok{)) }\SpecialCharTok{+} \FunctionTok{sum}\NormalTok{(r }\SpecialCharTok{*} \FunctionTok{log}\NormalTok{(p)) }\SpecialCharTok{+} \FunctionTok{sum}\NormalTok{((n}\SpecialCharTok{{-}}\NormalTok{r) }\SpecialCharTok{*} \FunctionTok{log}\NormalTok{(}\DecValTok{1} \SpecialCharTok{{-}}\NormalTok{ p))}
\NormalTok{    acc\_prob }\OtherTok{=} \FunctionTok{exp}\NormalTok{(top }\SpecialCharTok{{-}}\NormalTok{ bottom) }\CommentTok{\# Ratio des noyaux vaut 1 (symétrie du noyau)}
    
    \ControlFlowTok{if}\NormalTok{ (}\FunctionTok{runif}\NormalTok{(}\DecValTok{1}\NormalTok{) }\SpecialCharTok{\textless{}} \FunctionTok{min}\NormalTok{(}\DecValTok{1}\NormalTok{, acc\_prob))\{}
\NormalTok{      alpha0 }\OtherTok{=}\NormalTok{ prop}
\NormalTok{      p }\OtherTok{=}\NormalTok{ prop\_p}
\NormalTok{      acc\_rates[}\DecValTok{1}\NormalTok{] }\OtherTok{=}\NormalTok{ acc\_rates[}\DecValTok{1}\NormalTok{] }\SpecialCharTok{+} \DecValTok{1}
\NormalTok{    \}}
    
    \CommentTok{\# Mise à jour de alpha1}
\NormalTok{    alpha1 }\OtherTok{=}\NormalTok{ res[i, }\DecValTok{2}\NormalTok{]}
\NormalTok{    prop }\OtherTok{=} \FunctionTok{rnorm}\NormalTok{(}\DecValTok{1}\NormalTok{, alpha1, prop\_sd[}\DecValTok{2}\NormalTok{]) }\CommentTok{\# marche aléatoire simple}
\NormalTok{    prop\_p }\OtherTok{=} \FunctionTok{plogis}\NormalTok{(alpha0 }\SpecialCharTok{+}\NormalTok{ prop }\SpecialCharTok{*}\NormalTok{ x1 }\SpecialCharTok{+}\NormalTok{ alpha2 }\SpecialCharTok{*}\NormalTok{ x2 }\SpecialCharTok{+}\NormalTok{ alpha12 }\SpecialCharTok{*}\NormalTok{ x1 }\SpecialCharTok{*}\NormalTok{ x2 }\SpecialCharTok{+}\NormalTok{ b)}
    
\NormalTok{    top }\OtherTok{=} \SpecialCharTok{{-}}\NormalTok{ ((prop}\SpecialCharTok{\^{}}\DecValTok{2}\NormalTok{) }\SpecialCharTok{/}\NormalTok{ (}\DecValTok{2} \SpecialCharTok{*} \FloatTok{1e6}\NormalTok{)) }\SpecialCharTok{+} \FunctionTok{sum}\NormalTok{(r }\SpecialCharTok{*} \FunctionTok{log}\NormalTok{(prop\_p)) }\SpecialCharTok{+} \FunctionTok{sum}\NormalTok{((n}\SpecialCharTok{{-}}\NormalTok{r) }\SpecialCharTok{*} \FunctionTok{log}\NormalTok{(}\DecValTok{1} \SpecialCharTok{{-}}\NormalTok{ prop\_p))}
\NormalTok{    bottom }\OtherTok{=} \SpecialCharTok{{-}}\NormalTok{ ((alpha1}\SpecialCharTok{\^{}}\DecValTok{2}\NormalTok{) }\SpecialCharTok{/}\NormalTok{ (}\DecValTok{2} \SpecialCharTok{*} \FloatTok{1e6}\NormalTok{)) }\SpecialCharTok{+} \FunctionTok{sum}\NormalTok{(r }\SpecialCharTok{*} \FunctionTok{log}\NormalTok{(p)) }\SpecialCharTok{+} \FunctionTok{sum}\NormalTok{((n}\SpecialCharTok{{-}}\NormalTok{r) }\SpecialCharTok{*} \FunctionTok{log}\NormalTok{(}\DecValTok{1} \SpecialCharTok{{-}}\NormalTok{ p))}
\NormalTok{    acc\_prob }\OtherTok{=} \FunctionTok{exp}\NormalTok{(top }\SpecialCharTok{{-}}\NormalTok{ bottom) }\CommentTok{\# Ratio des noyaux vaut 1 (symétrie du noyau)}
    
    \ControlFlowTok{if}\NormalTok{ (}\FunctionTok{runif}\NormalTok{(}\DecValTok{1}\NormalTok{) }\SpecialCharTok{\textless{}} \FunctionTok{min}\NormalTok{(}\DecValTok{1}\NormalTok{, acc\_prob))\{}
\NormalTok{      alpha1 }\OtherTok{=}\NormalTok{ prop}
\NormalTok{      p }\OtherTok{=}\NormalTok{ prop\_p}
\NormalTok{      acc\_rates[}\DecValTok{2}\NormalTok{] }\OtherTok{=}\NormalTok{ acc\_rates[}\DecValTok{2}\NormalTok{] }\SpecialCharTok{+} \DecValTok{1}
\NormalTok{    \}}

    \CommentTok{\# Mise à jour de alpha12}
\NormalTok{    alpha12 }\OtherTok{=}\NormalTok{ res[i, }\DecValTok{3}\NormalTok{]}
\NormalTok{    prop }\OtherTok{=} \FunctionTok{rnorm}\NormalTok{(}\DecValTok{1}\NormalTok{, alpha12, prop\_sd[}\DecValTok{3}\NormalTok{]) }\CommentTok{\# marche aléatoire simple}
\NormalTok{    prop\_p }\OtherTok{=} \FunctionTok{plogis}\NormalTok{(alpha0 }\SpecialCharTok{+}\NormalTok{ alpha1 }\SpecialCharTok{*}\NormalTok{ x1 }\SpecialCharTok{+}\NormalTok{ alpha2 }\SpecialCharTok{*}\NormalTok{ x2 }\SpecialCharTok{+}\NormalTok{ prop }\SpecialCharTok{*}\NormalTok{ x1 }\SpecialCharTok{*}\NormalTok{ x2 }\SpecialCharTok{+}\NormalTok{ b)}
    
\NormalTok{    top }\OtherTok{=} \SpecialCharTok{{-}}\NormalTok{ ((prop}\SpecialCharTok{\^{}}\DecValTok{2}\NormalTok{) }\SpecialCharTok{/}\NormalTok{ (}\DecValTok{2} \SpecialCharTok{*} \FloatTok{1e6}\NormalTok{)) }\SpecialCharTok{+} \FunctionTok{sum}\NormalTok{(r }\SpecialCharTok{*} \FunctionTok{log}\NormalTok{(prop\_p)) }\SpecialCharTok{+} \FunctionTok{sum}\NormalTok{((n}\SpecialCharTok{{-}}\NormalTok{r) }\SpecialCharTok{*} \FunctionTok{log}\NormalTok{(}\DecValTok{1} \SpecialCharTok{{-}}\NormalTok{ prop\_p))}
\NormalTok{    bottom }\OtherTok{=} \SpecialCharTok{{-}}\NormalTok{ ((alpha12}\SpecialCharTok{\^{}}\DecValTok{2}\NormalTok{) }\SpecialCharTok{/}\NormalTok{ (}\DecValTok{2} \SpecialCharTok{*} \FloatTok{1e6}\NormalTok{)) }\SpecialCharTok{+} \FunctionTok{sum}\NormalTok{(r }\SpecialCharTok{*} \FunctionTok{log}\NormalTok{(p)) }\SpecialCharTok{+} \FunctionTok{sum}\NormalTok{((n}\SpecialCharTok{{-}}\NormalTok{r) }\SpecialCharTok{*} \FunctionTok{log}\NormalTok{(}\DecValTok{1} \SpecialCharTok{{-}}\NormalTok{ p))}
\NormalTok{    acc\_prob }\OtherTok{=} \FunctionTok{exp}\NormalTok{(top }\SpecialCharTok{{-}}\NormalTok{ bottom) }\CommentTok{\# Ratio des noyaux vaut 1 (symétrie du noyau)}
    
    \ControlFlowTok{if}\NormalTok{ (}\FunctionTok{runif}\NormalTok{(}\DecValTok{1}\NormalTok{) }\SpecialCharTok{\textless{}} \FunctionTok{min}\NormalTok{(}\DecValTok{1}\NormalTok{, acc\_prob))\{}
\NormalTok{      alpha12 }\OtherTok{=}\NormalTok{ prop}
\NormalTok{      p }\OtherTok{=}\NormalTok{ prop\_p}
\NormalTok{      acc\_rates[}\DecValTok{3}\NormalTok{] }\OtherTok{=}\NormalTok{ acc\_rates[}\DecValTok{3}\NormalTok{] }\SpecialCharTok{+} \DecValTok{1}
\NormalTok{    \}}
    
    \CommentTok{\# Mise à jour de alpha2}
\NormalTok{    alpha2 }\OtherTok{=}\NormalTok{ res[i, }\DecValTok{4}\NormalTok{]}
\NormalTok{    prop }\OtherTok{=} \FunctionTok{rnorm}\NormalTok{(}\DecValTok{1}\NormalTok{, alpha2, prop\_sd[}\DecValTok{4}\NormalTok{]) }\CommentTok{\# marche aléatoire simple}
\NormalTok{    prop\_p }\OtherTok{=} \FunctionTok{plogis}\NormalTok{(alpha0 }\SpecialCharTok{+}\NormalTok{ alpha1 }\SpecialCharTok{*}\NormalTok{ x1 }\SpecialCharTok{+}\NormalTok{ prop }\SpecialCharTok{*}\NormalTok{ x2 }\SpecialCharTok{+}\NormalTok{ alpha12 }\SpecialCharTok{*}\NormalTok{ x1 }\SpecialCharTok{*}\NormalTok{ x2 }\SpecialCharTok{+}\NormalTok{ b)}
    
\NormalTok{    top }\OtherTok{=} \SpecialCharTok{{-}}\NormalTok{ ((prop}\SpecialCharTok{\^{}}\DecValTok{2}\NormalTok{) }\SpecialCharTok{/}\NormalTok{ (}\DecValTok{2} \SpecialCharTok{*} \FloatTok{1e6}\NormalTok{)) }\SpecialCharTok{+} \FunctionTok{sum}\NormalTok{(r }\SpecialCharTok{*} \FunctionTok{log}\NormalTok{(prop\_p)) }\SpecialCharTok{+} \FunctionTok{sum}\NormalTok{((n}\SpecialCharTok{{-}}\NormalTok{r) }\SpecialCharTok{*} \FunctionTok{log}\NormalTok{(}\DecValTok{1} \SpecialCharTok{{-}}\NormalTok{ prop\_p))}
\NormalTok{    bottom }\OtherTok{=} \SpecialCharTok{{-}}\NormalTok{ ((alpha2}\SpecialCharTok{\^{}}\DecValTok{2}\NormalTok{) }\SpecialCharTok{/}\NormalTok{ (}\DecValTok{2} \SpecialCharTok{*} \FloatTok{1e6}\NormalTok{)) }\SpecialCharTok{+} \FunctionTok{sum}\NormalTok{(r }\SpecialCharTok{*} \FunctionTok{log}\NormalTok{(p)) }\SpecialCharTok{+} \FunctionTok{sum}\NormalTok{((n}\SpecialCharTok{{-}}\NormalTok{r) }\SpecialCharTok{*} \FunctionTok{log}\NormalTok{(}\DecValTok{1} \SpecialCharTok{{-}}\NormalTok{ p))}
\NormalTok{    acc\_prob }\OtherTok{=} \FunctionTok{exp}\NormalTok{(top }\SpecialCharTok{{-}}\NormalTok{ bottom) }\CommentTok{\# Ratio des noyaux vaut 1 (symétrie du noyau)}
    
    \ControlFlowTok{if}\NormalTok{ (}\FunctionTok{runif}\NormalTok{(}\DecValTok{1}\NormalTok{) }\SpecialCharTok{\textless{}} \FunctionTok{min}\NormalTok{(}\DecValTok{1}\NormalTok{, acc\_prob))\{}
\NormalTok{      alpha2 }\OtherTok{=}\NormalTok{ prop}
\NormalTok{      p }\OtherTok{=}\NormalTok{ prop\_p}
\NormalTok{      acc\_rates[}\DecValTok{4}\NormalTok{] }\OtherTok{=}\NormalTok{ acc\_rates[}\DecValTok{4}\NormalTok{] }\SpecialCharTok{+} \DecValTok{1}
\NormalTok{    \}}
    
    \CommentTok{\# Mise à jour de tau}
\NormalTok{    tau }\OtherTok{=} \FunctionTok{rgamma}\NormalTok{(}\DecValTok{1}\NormalTok{, }\AttributeTok{shape =} \FloatTok{10e{-}3} \SpecialCharTok{+}\NormalTok{ N }\SpecialCharTok{/} \DecValTok{2}\NormalTok{, }\AttributeTok{scale =} \FloatTok{1e{-}3} \SpecialCharTok{+} \FloatTok{0.5} \SpecialCharTok{*} \FunctionTok{sum}\NormalTok{(b}\SpecialCharTok{\^{}}\DecValTok{2}\NormalTok{))}

    \CommentTok{\# Mise à jour de b}
    \ControlFlowTok{for}\NormalTok{ (j }\ControlFlowTok{in} \DecValTok{1}\SpecialCharTok{:}\NormalTok{N)\{}
\NormalTok{      prop }\OtherTok{=} \FunctionTok{rnorm}\NormalTok{(}\DecValTok{1}\NormalTok{, b[j], prop\_sd[}\DecValTok{5}\NormalTok{])}
\NormalTok{      prop\_p\_j }\OtherTok{=} \FunctionTok{plogis}\NormalTok{(alpha0 }\SpecialCharTok{+}\NormalTok{ alpha1 }\SpecialCharTok{*}\NormalTok{ x1[j] }\SpecialCharTok{+}\NormalTok{ alpha2 }\SpecialCharTok{*}\NormalTok{ x2[j] }\SpecialCharTok{+}\NormalTok{ alpha12 }\SpecialCharTok{*}\NormalTok{ x1[j] }\SpecialCharTok{*}\NormalTok{ x2[j] }\SpecialCharTok{+}\NormalTok{ prop)}
      
\NormalTok{      top }\OtherTok{=} \SpecialCharTok{{-}}\NormalTok{ (prop}\SpecialCharTok{\^{}}\DecValTok{2} \SpecialCharTok{*}\NormalTok{ tau }\SpecialCharTok{/} \DecValTok{2}\NormalTok{) }\SpecialCharTok{+}\NormalTok{ r[j] }\SpecialCharTok{*} \FunctionTok{log}\NormalTok{(prop\_p\_j) }\SpecialCharTok{+}\NormalTok{ (n[j] }\SpecialCharTok{{-}}\NormalTok{ r[j]) }\SpecialCharTok{*} \FunctionTok{log}\NormalTok{(}\DecValTok{1} \SpecialCharTok{{-}}\NormalTok{ prop\_p\_j)}
\NormalTok{      bottom }\OtherTok{=} \SpecialCharTok{{-}}\NormalTok{ (b[j]}\SpecialCharTok{\^{}}\DecValTok{2} \SpecialCharTok{*}\NormalTok{ tau }\SpecialCharTok{/} \DecValTok{2}\NormalTok{) }\SpecialCharTok{+}\NormalTok{ r[j] }\SpecialCharTok{*} \FunctionTok{log}\NormalTok{(p[j]) }\SpecialCharTok{+}\NormalTok{ (n[j] }\SpecialCharTok{{-}}\NormalTok{ r[j]) }\SpecialCharTok{*} \FunctionTok{log}\NormalTok{(}\DecValTok{1} \SpecialCharTok{{-}}\NormalTok{ p[j])}
\NormalTok{      acc\_prob }\OtherTok{=} \FunctionTok{exp}\NormalTok{(top }\SpecialCharTok{{-}}\NormalTok{ bottom)}
      
      \ControlFlowTok{if}\NormalTok{ (}\FunctionTok{runif}\NormalTok{(}\DecValTok{1}\NormalTok{) }\SpecialCharTok{\textless{}} \FunctionTok{min}\NormalTok{(}\DecValTok{1}\NormalTok{, acc\_prob))\{}
\NormalTok{        b[j] }\OtherTok{=}\NormalTok{ prop}
\NormalTok{        p[j] }\OtherTok{=}\NormalTok{ prop\_p\_j}
\NormalTok{      \}}
\NormalTok{    \}}
    
    \CommentTok{\# Mise à jour de la chaine de Markov et de b}
\NormalTok{    res[i}\SpecialCharTok{+}\DecValTok{1}\NormalTok{, ] }\OtherTok{=} \FunctionTok{c}\NormalTok{(alpha0, alpha1, alpha12, alpha2, }\DecValTok{1}\SpecialCharTok{/}\FunctionTok{sqrt}\NormalTok{(tau))}
\NormalTok{    res\_b[i}\SpecialCharTok{+}\DecValTok{1}\NormalTok{, ] }\OtherTok{=}\NormalTok{ b}
\NormalTok{  \}}
  
\NormalTok{  my\_list }\OtherTok{\textless{}{-}} \FunctionTok{list}\NormalTok{(}\StringTok{"chain"} \OtherTok{=}\NormalTok{ res, }\StringTok{"b\_chain"} \OtherTok{=}\NormalTok{ res\_b, }\StringTok{"acc\_rates"} \OtherTok{=}\NormalTok{ acc\_rates)}
  \FunctionTok{return}\NormalTok{(my\_list)}
\NormalTok{\}}
\end{Highlighting}
\end{Shaded}

\hypertarget{ruxe9cupuxe9ration-des-ruxe9sultats}{%
\section{Récupération des
résultats}\label{ruxe9cupuxe9ration-des-ruxe9sultats}}

\begin{Shaded}
\begin{Highlighting}[]
\NormalTok{resultat }\OtherTok{=} \FunctionTok{seeds}\NormalTok{(}\FloatTok{1e4}\NormalTok{, N, r, n, x1, x2, alpha0, alpha1, alpha2, alpha12, tau, b, }\AttributeTok{prop\_sd =} \FunctionTok{c}\NormalTok{(}\FloatTok{0.3}\NormalTok{, }\FloatTok{0.3}\NormalTok{, }\FloatTok{0.3}\NormalTok{, }\FloatTok{0.3}\NormalTok{, }\FloatTok{0.3}\NormalTok{)) }\CommentTok{\# prop\_sd choisi pour avoir une allure de chaîne cohérente}

\NormalTok{resultat\_chain }\OtherTok{=}\NormalTok{ resultat}\SpecialCharTok{$}\NormalTok{chain[}\DecValTok{1001}\SpecialCharTok{:}\FunctionTok{nrow}\NormalTok{(resultat}\SpecialCharTok{$}\NormalTok{chain), ] }\CommentTok{\# on enlève les 1000 premiers (burnin)}

\NormalTok{moychain }\OtherTok{=} \FunctionTok{colMeans}\NormalTok{(resultat\_chain) }\CommentTok{\# moyenne}
\NormalTok{sdchain }\OtherTok{=} \FunctionTok{apply}\NormalTok{(resultat\_chain, }\DecValTok{2}\NormalTok{, sd) }\CommentTok{\# écart type}

\FunctionTok{cat}\NormalTok{(}\StringTok{"alpha\_0 est estimé à"}\NormalTok{, moychain[}\DecValTok{1}\NormalTok{], }\StringTok{"avec un écart{-}type de"}\NormalTok{, sdchain[}\DecValTok{1}\NormalTok{], }\StringTok{"}\SpecialCharTok{\textbackslash{}n}\StringTok{"}\NormalTok{)}
\end{Highlighting}
\end{Shaded}

\begin{verbatim}
## alpha_0 est estimé à -0.5117202 avec un écart-type de 0.1817535
\end{verbatim}

\begin{Shaded}
\begin{Highlighting}[]
\FunctionTok{cat}\NormalTok{(}\StringTok{"alpha\_1 est estimé à"}\NormalTok{, moychain[}\DecValTok{2}\NormalTok{], }\StringTok{"avec un écart{-}type de"}\NormalTok{, sdchain[}\DecValTok{2}\NormalTok{], }\StringTok{"}\SpecialCharTok{\textbackslash{}n}\StringTok{"}\NormalTok{)}
\end{Highlighting}
\end{Shaded}

\begin{verbatim}
## alpha_1 est estimé à 0.01711611 avec un écart-type de 0.2920292
\end{verbatim}

\begin{Shaded}
\begin{Highlighting}[]
\FunctionTok{cat}\NormalTok{(}\StringTok{"alpha\_12 est estimé à"}\NormalTok{, moychain[}\DecValTok{3}\NormalTok{], }\StringTok{"avec un écart{-}type de"}\NormalTok{, sdchain[}\DecValTok{3}\NormalTok{], }\StringTok{"}\SpecialCharTok{\textbackslash{}n}\StringTok{"}\NormalTok{)}
\end{Highlighting}
\end{Shaded}

\begin{verbatim}
## alpha_12 est estimé à -0.7781519 avec un écart-type de 0.3964975
\end{verbatim}

\begin{Shaded}
\begin{Highlighting}[]
\FunctionTok{cat}\NormalTok{(}\StringTok{"alpha\_2 est estimé à"}\NormalTok{, moychain[}\DecValTok{4}\NormalTok{], }\StringTok{"avec un écart{-}type de"}\NormalTok{, sdchain[}\DecValTok{4}\NormalTok{], }\StringTok{"}\SpecialCharTok{\textbackslash{}n}\StringTok{"}\NormalTok{)}
\end{Highlighting}
\end{Shaded}

\begin{verbatim}
## alpha_2 est estimé à 1.31412 avec un écart-type de 0.2552007
\end{verbatim}

\begin{Shaded}
\begin{Highlighting}[]
\FunctionTok{cat}\NormalTok{(}\StringTok{"sigma est estimé à"}\NormalTok{, moychain[}\DecValTok{5}\NormalTok{], }\StringTok{"avec un écart{-}type de"}\NormalTok{, sdchain[}\DecValTok{5}\NormalTok{], }\StringTok{"}\SpecialCharTok{\textbackslash{}n}\StringTok{"}\NormalTok{)}
\end{Highlighting}
\end{Shaded}

\begin{verbatim}
## sigma est estimé à 0.3203446 avec un écart-type de 0.0666796
\end{verbatim}

\end{document}
